\section{Conditional Statements}
Conditional statements are part of program that are executed depending whether certain condition holds. In Smalltalk, a condition statement is a sequence of messages that are only sent  when a boolean expression is in a given state. We will use the following convention we say that a conditional statement or expression is composed by a \textit{boolean expression} which specifies the condition, and some \textit{conditional code}. We say that  the conditional code from a \textit{branch} because the program execution can follow or not a certain part of the program forming a kind of tree.

Smalltalk offers four different methods to express condition: \index{ifTrue:}  \ct{ifTrue:}, \index{ifFalse:}  \ct{ifFalse:} \index{ifTrue:ifFalse:}\ct{ifTrue:ifFalse:}, and \index{ifFalse:ifTrue:}  \ct{ifFalse:ifTrue:}. However there are in fact two main variants  depending whether there is one (\ct{ifTrue:} and \ct{ifFalse:}) or two branches in the conditions (\ct{ifFalse:ifTrue:} and \ct{ifTrue:ifFalse:}). 

\subsection{The Messages \ct{ifTrue:} and \ct{ifFalse:}}
\paragraph{The message \index{ifTrue:}\ct{ifTrue:}.} The message \ct{aBooleanExpression ifTrue: aBlock} specifies that  the sequence of messages, \ct{aBlock} will only be executed whether the expression \ct{aBooleanExpression} is \textit{true}. 

\begin{scriptwithtitle}{Example of \ct{ifTrue:}}\label{scr:iftrueExample}
Time now > (Time new hours: 13)
   \textbf{ifTrue: [PopUpMenu inform: 'Time for a nap']}.
Smalltalk beep. 
\end{scriptwithtitle}

In \tscrref{scr:iftrueExample} the boolean expression \ct{Time now > (Time new hours: 13)} is evaluated. 
It will only be true when it will be  later than 13 o'clock. In such a case and only in such a case, the conditional code, \ct{ [PopUpMenu inform: 'Time for a nap']}, is executed. Here, it  shows the message 'Time for a nap'. When the boolean expression is false, the condition is not executed at all. Note that the code after the condition is executed as normal. Here the expression \ct{Smalltalk beep} is always executed because it does not belongs to the conditional code. Therefore its execution is not influenced by the value of the boolean expression. 

\paragraph{The message\index{ifFalse:}\ct{ifFalse:}.} The message \ct{ifFalse:} is similar than the message \ct{ifTrue:} except that it will only execute the conditional code whether the boolean expression is false. 
The message \ct{aBooleanExpression ifFalse: aBlock} specifies that  the sequence of messages, \ct{aBlock}, will only be executed whether the expression \ct{aBooleanExpression} is false. 

We can express \tscrref{scr:iftrueExample} using \ct{ifFalse:} by simply negating the boolean expression 
as shown by the scripts~ref{scr:iffalseExample} or \ref{scr:iffalseExample2}. In \tscrref{scr:iffalseExample} we negated the expression using not while in \tscrref{scr:iffalseExample2} we changed the expression. 

\begin{scriptwithtitle}{Example of \ct{ifTrue:}}\label{scr:iffalseExample}
(Time now > (Time new hours: 13)) not
   ifFalse: [PopUpMenu inform: 'Time for a nap'].
Smalltalk beep. 
\end{scriptwithtitle}

\begin{scriptwithtitle}{Example of \ct{ifTrue:}}\label{scr:iffalseExample2}
(Time now <= (Time new hours: 13)) 
   ifFalse: [PopUpMenu inform: 'Time for a nap'].
Smalltalk beep. 
\end{scriptwithtitle}


\begin{template}
\textit{aCondition}
   \textbf{ifTrue: [} \textit{messagesIfConditionIsTrue} \textbf{]}

\textit{aCondition}
   \textbf{ifFalse: [} \textit{messagesIfConditionIsFalse} \textbf{]}
\end{template}


\subsection{The message  \ct{ifTrue:ifFalse:} and  \ct{ifFalse:ifTrue:}}

The message \ct{aBooleanExpression ifTrue: aTrueBlock ifFalse: aFalseBlock} specifies that  the sequence of messages, \ct{aTrueBlock} will only be executed whether the expression \ct{aBooleanExpression} is true and that the the sequence of messages, \ct{aFalseBlock} will only be executed whether the expression \ct{aBooleanExpression} is false. In short \ct{aTrueBlock} is executed if the \ct{booleanExpression} is true else \ct{aFalseBlock} is executed. In any case, \ct{aTrueBlock} and \ct{aFalseBlock} cannot be executed both. 


\begin{scriptwithtitle}{Example of \ct{ifTrue:ifFalse:}}\label{scr:iftruefalseExample}
Time now > (Time new hours: 13)
   ifTrue: [PopUpMenu inform: 'Time for a nap']
   ifFalse: [Smalltalk beep]. 
\end{scriptwithtitle}

The behavior of \tscrref{scr:iftruefalseExample} is different than \tscrref{scr:iftrueExample} in the sense that only one of the branches will be executed and not both. Here either we will obtain the message or the sound, never both.


As both banches are under the same condition, the method \ct{ifFalse:ifTrue:} is just a convenient method
to write the code in the order we prefer. Contrary to the difference we did when passingfrom \ct{ifTrue:} (\scrref{scr:iftrueExample}to \ct{ifFalse:} \scrref{scr:iffalseExample}), to use \ct{ifTrue:ifFalse:} or \ct{ifFalse:ifTrue:} we do not have to change anything because both cases are already expressed as shown by~\tscrref{scr:falseiftrueExample}. We just permute the conditions.

\begin{scriptwithtitle}{Example of \ct{ifTrue:ifFalse:}}\label{scr:falseiftrueExample}
Time now > (Time new hours: 13)
   ifFalse: [Smalltalk beep]
   ifTrue: [PopUpMenu inform: 'Time for a nap']. 
\end{scriptwithtitle}