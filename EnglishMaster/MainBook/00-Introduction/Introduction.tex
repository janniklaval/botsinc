% $Author: stef $
% $Date: 2008-04-04 17:14:31 +0200 (Fri, 04 Apr 2008) $
% $Revision: 318 $
%=================================================================
\ifx\wholebook\relax\else
% --------------------------------------------
% Lulu:
    \documentclass[a4paper,10pt,twoside]{book}
    \usepackage[
        papersize={6in,9in},
        hmargin={.75in,.75in},
        vmargin={.75in,1in},
        ignoreheadfoot
    ]{geometry}
    \input{../common.tex}
    \pagestyle{headings}
    \setboolean{lulu}{true}
% --------------------------------------------
% A4:
%   \documentclass[a4paper,11pt,twoside]{book}
%   \input{../common.tex}
%   \usepackage{a4wide}
% --------------------------------------------
    \graphicspath{{figures/} {../figures/}}
    \begin{document}
%   \renewcommand{\nnbb}[2]{} % Disable editorial comments
    \sloppy
\fi

\chapter{Preface}\label{cha:preface}

\begin{quote}
\textit{Knowledge is only one part of understanding.Genuine understanding comes 
from hands-on experience.}
—S. Papert 
\end{quote}


\section*{Goals and Audience}
The goal of this book is to explain elementary programming concepts (such as loops, abstraction, 
composition, and conditionals) to novices of all ages. I believe that learning by experimenting 
and solving problems is central to human knowledge acquisition. Therefore, I have presented 
programming concepts through simple problems such as drawing golden rectangles or simulat- 
ing animal behavior. 

My ultimate goal is to teach you object-oriented programming, because this particular 
paradigm provides an excellent metaphor for teaching programming. However, object- 
oriented programming requires some more elementary notions of programming and abstrac- 
tion. Therefore, I wrote this book to present these basic programming concepts in an elementary 
programming environment with the special perspective that this book is the first in a series of 
two books. Nevertheless, this book is completely self-contained and does not require you to 
read the next one. The second book introduces another small programming environment. It 
focuses on intermediate-level topics such as finding a path through a maze and drawing frac- 
tals. It also acts as a companion book for readers who want to know more and who want to 
adapt the environment of this book to their own needs. Finally, it introduces object-oriented 
programming. 

The ideal reader I have in mind is an individual who wants to have fun programming. 
This person may be a teenager or an adult, a schoolteacher, or somebody teaching program- 
ming to children in some other organization. Such an individual does not have to be fluent in 
programming in any language. 

The material of this book was originally developed for my wife, who is a physics and 
mathematics teacher in a French school where the students are between eleven and fifteen 
years old. In late 1998, my wife was asked to teach computing science, and she was dismayed 
by the lack of appropriate material. She started out teaching HTML, Word, and other topics, 
and she remained dissatisfied, since these approaches failed to promote a scientific attitude 
toward computing science. Her goal was to teach computer science as a process of attacking 
problems and finding solutions. 

As a computer scientist, I was aware of work on the programming language Logo, and I 
particularly liked the idea of experimentation as a basis for learning. I was also aware that the 
programming language Smalltalk had been influenced by the ideas of Seymour Papert and 
those behind Logo, and that it had originated from research on teaching programming to 
children. Moreover, Smalltalk has a simple syntax that mimics natural language. At about that 
time, the Squeak environment had arrived at a mature state, and books started to become 
available in late 1999. But these were for experienced programmers, so I started and wrote 
the present book. 

The environments that I use in this book and its companion book are fully functional. 
They have gone through several iterations of improvements based on the feedback that I have 
received from teachers. A guiding rule in our work has been to modify the Squeak environment 
as little as possible, for our goal is for readers to be able to extend the ideas presented in this 
book and develop new ones of their own. 

\section*{Object-Oriented Structure and Vocabulary}

The chapters of this book are relatively small, so that each chapter can be turned into a one- 
or two-hour lab session. I do not advocate presenting the material directly to children for self- 
instruction, but each chapter in fact has all the material for such an approach. 
Although object-oriented programming is not developed in this book, I use its vocabu- 
lary. That is, we create objects from classes and send them messages. Object behavior is 
defined by methods. I made this choice because the metaphor offered by object-oriented pro- 
gramming is natural, and children have an intuitive understanding of the idea of objects and 
their behavior. 

Those who are used to Logo may wonder why our robots do not have “pen up” and “pen 
down” methods, but instead “go” and “jump,” where under the former, a robot moves leaving 
a trace, while the latter moves a robot forward without leaving a trace. I believe that the go and 
jump paradigm is better suited to the ideas of object-oriented programming and encapsula- 
tion of data than the traditional pen down and pen up design. An excellent analysis of these 
two approaches was made by Didier Besset, who collaborated with me on this project in its 
early stages. 


\section*{Organization}


The book is divided into five parts, as described below. 

\begin{description}
	\item Getting Started. Part 1 shows how to get started with the Squeak environment. It explains 
the installation process and how to launch Squeak, and then presents robots and their 
behavior. A first simple program that draws some lines is presented. 
	\begin{center}
	\includegraphics[width=5cm]{1-ChTurntitlePicture}
	\end{center}
	
	\item Elementary Programming Concepts.Part 2 introduces first programming concepts, such 
	as loops and variables. It shows how messages sent to a robot are resolved. 
	
	\begin{center}
	\includegraphics[width=4cm]{2-varLoopsTitle}
	\end{center}
	
	\item Bringing Abstraction into Play. Part 3 introduces the necessity of abstraction, that is, 
	methods or procedures that can be reused by different programs. The most difficult con- 
	cept introduced is the idea of composing new methods from existing ones to solve more 
	complex problems. Several nontrivial experiments are proposed, such as drawing golden 
	rectangles. Techniques and tools for debugging programs are also introduced. 
	
	
	\begin{center}
	\includegraphics[width=5cm]{3-nborsteps}
	\end{center}
	
	\item Conditionals. Part 4 introduces the notion of conditionals, conditional loops, and Boolean 
	expressions, all of which are central to programming. This part also introduces the notion of 
	references in a two-dimensional space and some other types of robot behavior. Finally, 
	ways of using robots to simulate the behavior of simple animals are presented. 
	
	\begin{center}
	\includegraphics[width=3cm]{4-followBorder2}\includegraphics[width=3cm]{5-oppositeBorderOfBox}
	\end{center}


\item Other Squeak Worlds. Part 5 presents two entertaining programming environments that 
are available in Squeak: the eToy graphical scripting system and the 3D authoring envi- 
ronment Alice. 

\end{description}

\section*{Why Squeak and Smalltalk?}
You may be wondering why among the large number of programming languages available today 
I have chosen Smalltalk. Smalltalk and Squeak have been chosen for the following reasons: 

\begin{itemize}
\item  Smalltalk is a powerful language. You can build extremely complex systems within a 
language that is simple and uniform. 
\item  Smalltalk was designed as a teaching language. It was influenced by Logo and LISP, and 
Smalltalk in turn heavily influenced languages such as Java and C\#. However, those lan- 
guages are much too complex for a first exposure to programming. They have lost the 
beauty of Smalltalk’s simplicity. 
\item  Smalltalk is dynamically typed, and this makes transparent a number of concerns 
related to types and type coercion that are tedious to explain and of little interest to 
the novice. 
\item  With Smalltalk you need to learn only key, essential concepts, concepts that are to be 
found in all programming languages. Thus with Smalltalk I can focus on explaining the 
important concepts without having to deal with difficult or unattractive aspects of 
more complex languages. 
\item  Squeak is a powerful multimedia environment, so after reading my books you will be 
able to build your own programs in a truly rich context. 
\item  Squeak is available without charge and runs on all of today’s principal computing plat- 
forms. And it should be easily portable to the platforms of the future. 
\item  Squeak is popular. For example, in Spain, it is used in schools, where it runs on over 
80,000 computers.
\end{itemize}

\ifx\wholebook\relax\else
    \end{document}
\fi

%%% Local Variables:
%%% coding: utf-8
%%% mode: latex
%%% TeX-master: t
%%% TeX-PDF-mode: t
%%% ispell-local-dictionary: "english"
%%% End:
